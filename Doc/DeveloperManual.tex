\documentclass{article}
\usepackage{fancyvrb}
\usepackage{listings}
\lstset{
basicstyle=\small\ttfamily,
columns=flexible,
breaklines=true
}
\let\oldsection\section
%\renewcommand\section{\clearpage\oldsection}

\begin{document}
\title{ALPIDE Characterisation Software Framework - Manual}
\date{rev. 1, \today}

\maketitle


\section{Introduction}

The software framework described in this document offers a lean
software environment, which facilitates the development of test
routines for the ALPIDE chip and ALPIDE modules.

The guidelines for the architecture development were the following: 

\begin{itemize}
\item Simplicity of application development: the framework offers
  a package of building blocks that allow the user to easily implement
  applications for tests. Each test is supposed to be implemented in
  its own small standalone application.
\item Independence from the readout system: as far as the generic
  functions of the readout boards are concerned, the implementation
  is transparent to the application, i.e. the high level
  implementation does not depend on which readout board is used.  
\item As a consequence the same Alpide implementation and (within
  limits) the same test routines can be used with all readout cards /
  test systems.
\end{itemize}


These requirements were addressed by the following class structure:
\begin{itemize}
\item An abstract base class TReadoutBoard, which all readout boards
  are derived from. Common functionality is declared in TReadoutBoard,
  special functionality of the different cards in the derived
  classes. 
\item A class TAlpide containing the basic chip interface and the
  register map. 
\item Two helper classes TAlpideDecoder and TAlpideConfig for event
  decoding and standard configuration commands, resp.
\item A class TConfig containing configuration information.
\end{itemize}

This document describes how the framework can be used to implement
test applications. The first part of the
document gives a step-to-step guide to the application development,
the last part serves as a reference for the implemented methods. 

\section {Manual}

\subsection{Initialisation of the Setup}

The initialisation of the setup is the only part of each application,
which necessarily depends on the type of readout board used. In
general the initialisation consists of three steps (four in case of
the DAQ board):
\begin{enumerate}
\item Generate a config object, which contains all information on the
  type of the setup and the settings of chip(s) and readout board(s). 
\item Create a readout board object.
\item Create the chip object(s) and the connections between chip and
  board. The latter consists in passing the chip object a pointer to
  the readout board and ``telling'' the readout board object, which
  chip (identified via the chip ID) is connected to which control
  interface and which data receiver (non-trivial only for composed
  structures modules, HICs, staves). 
\item For the DAQ board: power on the chip.  
\end{enumerate}

A standard initialisation for a setup with one DAQ board and a single
chip with ID 16 is given in the following example (detailed
explanations of the used methods in the reference part): 

\begin{lstlisting}

// vectors containing all chips and boards 
// (in this example only one)
std::vector <TReadoutBoard *> fBoards;
std::vector <TAlpide *>       fChips;
TConfig                      *fConfig;

// create default config for one chip with ID 16
fConfig = new TConfig (16);
 
// USB helper-methods that create a DAQ board and push it into fBoards
InitLibUsb    ();
FindDAQBoards (fConfig, fBoards);
  
if (boards.size() != 1) { 
  std::cout << "Problem in finding DAQ board" << std::endl;
  // throw exception
}

// create the chip object and the connections board <-> chip
fChips. push_back(new TAlpide (config->GetChipConfig(16));
fChips. at(0) -> SetReadoutBoard (boards.at(0));
fBoards.at(0) -> AddChip         (16, 0, 0);

// DAQ-board specific instructions  
TReadoutBoardDAQ *myDAQBoard = dynamic_cast<TReadoutBoardDAQ*> (boards.at(0));
int               overflow;

if (myDAQBoard) {     // otherwise board was not of type TReadoutBoardDAQ
  if (! myDAQBoard->PowerOn(overflow) ) {
    std::cout << "LDOs did not switch on" << std::endl;
    // throw exception
  }
}
\end{lstlisting}


Please also have a look at the  file \texttt{SetupHelpers.cpp} in the repository / software package where several initialisations for standard setup types (single chip, IB HIC, OB HIC, half-stave) are already implemented.

\subsection{Interaction with the Readout Board}

The basic configuration of the readout board is done automatically
according to the information in the config, when the board object is
constructed (configuration for readout see below). For more
specialised configurations of the board  each readout board can be configured directly using the generic
\texttt{ReadRegister} and \texttt{WriteRegister} method or the
methods of the derived readout board classes (see list in
reference section). To access the methods of the derived classes the
readout board has to be cast on the appropriate class. Checks of the
object type can be done using a dynamic cast, e.g. 

\begin{lstlisting}
TReadoutBoardDAQ *myDAQBoard = dynamic_cast<TReadoutBoardDAQ*>(myReadoutBoard);
if (myDAQBoard) ... 
\end{lstlisting}

or alternatively with the method \texttt{GetBoardType()}.


\subsection{Chip Configuration}

The configuration of the chip can be done directly using the
methods \texttt{TAlpide::} \texttt{WriteRegister}, \texttt{TReadoutBoard::SendOpCode} and
\texttt{TAlpide::ReadRegister}. The register addresses can be given either
directly as integers or using the names defined in \texttt{TAlpide.h},
e.g. \texttt{Alpide::REG\_VCASN}. Additionally several methods for
standard configuration procedures (masking, CMU configuration...) have
been defined in AlpideConfig. All methods in AlpideConfig are used
with the chip object as parameter, e.g.
\begin{lstlisting}
// unmask all pixels
AlpideConfig::WritePixRegAll (myChip, Alpide::PIXREG_MASK, false);
// select row 5 for injection 
AlpideConfig::WritePixRegRow (myChip, Alpide::PIXREG_SELECT, true, 5);
\end{lstlisting}

A command sequencer that can be used to send a predefined series of
commands to the chip is also foreseen. 



\subsection {Data Taking}

In this framework data taking is separated from the data decoding,
giving the possibility to record either the raw data or hit
information. Two configuration methods exist to define the sequence of
pulse and strobe that is sent to the chip as well as the trigger
source. After configuration one method (\texttt{TReadoutBoard::Trigger (int
  nTriggers)}) can be used to send triggers to the chip
or module, a second one (\texttt{TReadoutBoard::ReadEventData()})
to retrieve the data event by event from the readout board objects. The two
methods can be called one after the other (for low number of triggers,
e.g. 50 triggers in a threshold scan) or can be put in two parallel
threads, which is recommended for larger number of triggers to avoid
filling the readout board buffer. 

Event decoding is performed by two (static) helper classes
\texttt{AlpideDecoder} and \texttt{BoardDecoder}, which decode the chip
event or the readout board header and trailer, resp. To do so the
buffer returned by \texttt{ReadEventData()} is first passed to
\texttt{BoardDecoder::DecodeEvent()}, then to
\texttt{AlpideDecoder::DecodeEvent()}. 



\paragraph{MOSAIC only:} In case of the MOSAIC board the \texttt{StartRun()} method has to be called before the data taking and \texttt{StopRun()} after.

Examples for the implementation of data taking can be seen in the files \texttt{main\_threshold.cpp, main\_digital.cpp, main\_noiseocc.cpp}



\subsection {Scan Class} 
For better portability of the scans to other softwares, in particular a GUI, a generic scan class \texttt{TScan} has been implemented. The class implements a generic scan consisting of three nested loops. The specific functionality performed in these loops can then be implemented in derived classes. Up to now this has been done and tested for the threshold scan. A sample application, which performs the threshold scan using the \texttt{TThresholdScan} class looks like this: 

\begin {lstlisting}
int main() {

  initSetup();

  TThresholdScan *myScan = new TThresholdScan(fConfig->GetScanConfig(), fChips, fBoards);

  myScan->Init();

  myScan->LoopStart(2);
  while (myScan->Loop(2)) {
    myScan->PrepareStep(2);
    myScan->LoopStart  (1);
    while (myScan->Loop(1)) {
      myScan->PrepareStep(1);
      myScan->LoopStart  (0);
      while (myScan->Loop(0)) {
        myScan->PrepareStep(0);
        myScan->Execute    ();
        myScan->Next       (0);  
      }
      myScan->LoopEnd(0);
      myScan->Next   (1);
    }
    myScan->LoopEnd(1);
    myScan->Next   (2);
  }
  myScan->LoopEnd  (2);
  myScan->Terminate();
  
  return 0;
}
\end{lstlisting}

Note that the implementation would be identical for other types of scans, with the exception of the call to the appropriate constructor. 

\clearpage
\section {Reference}

\subsection{Readout Board}

All readout board types used for the readout of ALPIDE chips or modules are
derived from the class TReadoutBoard. The following sections describe
the common functionality of all readout boards. (For the special
functionality of the different board types for the time being refer to
the header files \texttt{TReadoutBoardDAQ.h} and \texttt{TReadoutBoardMOSAIC.h})


\subsubsection{Board Communication}

\begin{itemize}
\item \texttt{int ReadRegister(uint16\_t Address, uint32\_t \&Value)}:
  \newline read value of readout board register at address \texttt{Address}.
\item \texttt{int WriteRegister(uint16\_t Address, uint32\_t Value)}:
  \newline write value \texttt{Value} into readout board register at address
\texttt{Address}.
\end{itemize}



\subsubsection{Chip Control Communication}

The methods for the chip control communication are supposed to be
called only through the chip object and are therefore protected. The
only exception is the method to read the chip registers, which can, if
needed, also be called directly through the readout board (e.g. in a
module where it is not clear, which chip addresses exist / function). 

\begin{itemize}
\item \texttt{int ReadChipRegister(uint16\_t Address, uint16\_t \&Value, uint8\_t ChipID = 0)}: \newline read value of chip register at address \texttt{Address}.
\end{itemize}


\subsubsection{Triggering, Pulsing and Readout}

Triggering and pulsing consists in sending a combination of first pulse, then strobe to the chip. Either one can be omitted (i.e. sending only pulse or only trigger) and the delays between them are programmable. Also the source for the trigger can be selected: 

\begin{itemize}
\item \texttt{void SetTriggerConfig (bool enablePulse, bool
  enableTrigger, int triggerDelay, int pulseDelay)}: Chose which
signal to send upon a trigger (software or external) as well as the
delays from pulse to strobe (\texttt{triggerDelay}) and from strobe to
the next pulse (\texttt{pulseDelay}, only relevant for MOSAIC).
\item \texttt{void SetTriggerSource (TTriggerSource triggerSource)}:
  chose between internal or external trigger, trigger-source type
  defined in \texttt{TReadoutBoard.h}.
\end{itemize}


For data taking itself the readout board provides the following two methods: 

\begin{itemize}
\item \texttt{int Trigger (int NTriggers)}: Sends \texttt{NTriggers}
  triggers to the chip(s). Here trigger means the strobe opcodes,
  pulse opcodes or a combination of pulse then strobe with the
  programmed distance inbetween. (The latter kept for redundancy,
  aim would be to send only pulses and generate the strobe on-chip.)
\item \texttt{int ReadEventData(int \&NBytes, char *Buffer)}: Returns
  the event data received by the readout card. Event data is in raw
  (undecoded) format, including readout card headers and
  trailers. The method returns one event at a time. Decoding is done in
  two separate decoder classes.\end{itemize}





\subsubsection{General}


\paragraph {Construction:}

TReadoutBoard is the abstract base class for all readout boards. A
readout board is therefore typically constructed the following way: 

\begin{lstlisting}
TReadoutBoard *myReadoutBoard = new TReadoutBoardDAQ   (config);
TReadoutBoard *mySecondBoard  = new TReadoutBoardMosaic(config);
\end{lstlisting}

\paragraph {Setup:}
In order for the control communication and the data readout to work in
all cases (single chips, IB staves and OB modules) the readout card
needs to have information on which chip, defined by its ID, is
connected to which control port and to which data receiver. This
information has to be added once from a configuration and then stored
internally, such that the chip ID is the only parameter for all
methods interacting with the chip for control or readout. The
necessary information is added by the method 

\begin{lstlisting}
int AddChip (uint8_t ChipID, int ControlInterface, int Receiver)
\end{lstlisting}



\subsection {Alpide Chip}
This class implements the interface of the alpide chip (control
interface and data readout) as well as the list of accessible register
addresses and will be used for all test setups. All further
information on the internal functionality of the chip are contained in
the helper classes AlpideConfig (for configuration information) and
AlpideDecoder (for decoding of event data). The class TAlpide
contains the following functions: 


\subsubsection {Constructing etc.}

\begin{itemize}
\item \texttt{TAlpide (TChipConfig *config)}: \newline
  Constructs TAlpide chip according to chip configuration.
\item \texttt{TAlpide (TConfig *config, TReadoutBoard *myROB)}: \newline Constructor
  including pointer to readout board
\item \texttt{void SetReadoutBoard(TReadoutBoard *myROB)}: \newline
  Setter function for readout board
\item \texttt{TReadoutBoard *GetReadoutBoard ())}: \newline (Private)
  Getter function for readout board
\end{itemize}


\subsubsection {Low Level Functions}

\begin{itemize}
\item \texttt{int ReadRegister(Alpide::TRegister Address, uint16\_t
    \&Value}: \newline read value of chip register at address
  \texttt{Address}. (A second version with an integer address
  exists).
\item \texttt{int WriteRegister(Alpide::TRegister Address, uint16\_t
    Value}: \newline write value \texttt{Value} into chip register at
  address (A second version with an integer address
  exists).
\texttt{Address}.
\item \texttt{int ModifyRegisterBits(Alpide::TRegister Address, int
    lowBit, int nBits, int Value)}: write bits [\texttt{lowBit, lowBit + nBits
  - 1}] of register \texttt{Address}. 
\item \texttt{int SendOpCode(uint8\_t OpCode)}: \newline Send an Opcode
  (NB: in case of modules this sends an opcode to all chips connected
  to the same command interface as the current chip.)
\end{itemize}




\subsubsection {Register Definitions} 

Chip registers are published in an enum type \texttt{TAlpideRegister}

\subsubsection{High Level Functions}

Higher level functionality of the alpide chip is implemented in two
helper classes: AlpideDecoder for the event decoding and
AlpideConfig for all configuration commands that go beyond mere
communication with the chip and act upon the internal functionality of
the chip.


\subsection{Event Decoding}

Event decoding is done in two static helper-classes (i.e. no object needs to be instantiated). Both provide a method \texttt{DecodeEvent}:

\begin{itemize}
\item \texttt{bool AlpideDecoder::DecodeEvent(unsigned char *data, int
    nBytes, std::vector <TPixHit> *hits)}: Tries to decode the event
  contained in \texttt{data} and stores the found hits in
  \texttt{hits}. Returns \texttt{true} in case of successful decoding,
  \texttt{false} otherwise. The pixel hit type is defined in
  \texttt{AlpideDecoder.h}.
\item \texttt{bool BoardDecoder::DecodeEvent(TBoardType boardType, unsigned
    char *data, int \&nBytes, TBoardHeader \&boardInfo)}: Tries to
  decode board header and trailer of the event contained in
  \texttt{data}. The complete information of header and trailer
  (flags, time stamps...) is stored in \texttt{boardInfo}; \texttt{data}
  and \texttt{nBytes} are reduced to the chip event only. The header
  type is defined in \texttt{BoardDecoder.h}, the board type in
  \texttt{TReadoutBoard.h}. 
\end{itemize}



\subsection{Config class}

The class TConfig contains all configuration information on the setup
(module, single chip, stave, type of readout board, number of chips)
as well as for the chips and the readout boards. It is based on /
similar to the TConfig class used by the software for the Cagliari
readout board and MATE. In addition to those versions it will allow
modification / creation on the fly by the software (Use case, e.g.:
create a config object after an automated check, which chips of the
module work; modify settings according to parameters entered in the
GUI by the user). 

The TConfig object, which is needed for the initialisation of the
setup can be constructed in three different ways: 

\begin{itemize}
\item \texttt{TConfig(const char *fName)}: reads the config from the
    given file.
\item \texttt{TConfig(int nBoards, std::vector <int> chipIds, TBoardType boardType = boardMOSAIC)}:
  creates a config for a given number of readout boards of type \texttt{boardType}, and chips with the
  given chip IDs. In this case standard settings are used as defined
  in the header files \texttt{TBoardConfig.h} and
  \texttt{TChipConfig.h}.
\item \texttt{TConfig(int chipId), TBoardType boardType = boardDAQ}: creates a standard config for one
  readout board of type \texttt{boardType} and one chip with the given chip Id. This is currently the
  only implemented constructor and can be used for initial testing of
  Alpide single chips.
\end{itemize}


\end{document} 




